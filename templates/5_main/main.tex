\chapter{Теоретичні відомості}


\section{Камера-обскури та стереопара}
В даній роботі ми маємо вхідні дані з двух камер-обскур. Камери-обскура --- 
це прототип фотографічного апарату, темне приміщення з одним малим отвором, 
через який на протилежну стіну проектується перевернуте зменшене зображення 
предметів ззовні (сцени). Цей малий отвір ми називаємо центром камери, а 
протилежна стіна є площиною отриманого зображення. Не втрачаючи загальності, 
перенесемо площину зпроектованого зображення до уявної площини між центром 
камери та об'єктом (сценою). Тоді отримане зображення не буде перевернутим.  

На вхід ми маємо пару плоских зображень одного і того ж об'єкта, зазначених вище.
Зображення мають відмінності між собою та покликані створити ефект об'єму. Ефект 
об'єму виникає в силу того, що розташовані на різній відстані від камери-обскури 
частини сцени (об'єкту) при фотографуванні з різних точок мають різне кутове 
зміщення (паралакс). Надалі ми будемо називати лівим --- зображення, центр камери
якого знаходиться лівіше в просторі відносно горизонтальної осі $х$, а правим ---
друге зображення, центр камери якого правіше. Така пара зображень називається 
стереопарою. Саме стереопара дається на вхід в нашій роботі.


\section{Сцена}
//Зображення сцени та фотоапаратів\\
Нехай точка $Х$ належить деякому об'єкту в трьохвимірному просторі. Ми маємо 
стереопару цього об'єкту. Тож ми маємо два зображення, де лежать проєкції точки 
$X$. Назвемо її проєкцію на ліве зображення --- $x$, а проєкцію на праве 
зображення --- $x'$. Зазначемо, що $x$ та $x'$ точки у трьохвимірному просторі. 
Введемо центри камер наших зображень: $O$ --- лівого та $O'$ --- правого. 
Очевидно, що пряма $Ox$ перетинає $X$, так само як і пряма $O'x'$. Разом прямі 
$Ox$ та $O'x'$ утворюють площину $\pi$. Якщо центри камер відомі, то $\pi$ може 
бути задана як і точкою $X$, так і точками $x$ чи $x'$. Таким чином для кожної
точки чи проекції точки деякого об'єкту, який ми фотографуємо, існує відповідна 
площина $\pi$, яка визначає певну точку об'єкта та дві її проекції на зображення 
стереопари. Далі будемо називати площину $\pi$ --- епіполярна площина точки $X$.


\section{Епіполярні лінії}
Центри камер $O$ та $O'$ --- фактично теж лежать у трьохвимірному просторі нашої
сцени. Отже вони теж проектуються на зображення. Назвемо проекцію $О$ на праве 
зображення --- $e'$, а проекцію $O'$ на ліве зображення --- $e$. Точки $e$ та 
$e'$ називаються лівою та правою епіполярними точками (епіполями) відповідно. 

Епіполі лежать на прямій $OO'$, отже вони теж належать площині $\pi$. Тепер, 
як зазначено вище, побудуємо $\pi$ через $x$ --- деяку проекцію на ліве 
зображення. Маємо питання: А де саме знаходиться точка $x'$ --- відповідна
проекція на праве зображення, відносно площини $\pi$?

Введемо точку $X$, яку ми проектували в $x$ та $x'$. Ми знаємо однозначно, що 
$X$ знаходиться на прямій $Ox$. Спроектуємо цю пряму на праве зображення. 
Оскільки центр правого зображення $O'$ лежить на площині $\pi$, то і проекція
прямої з $\pi$ теж буде лежати на площині $\pi$. Маємо, що проекція лежить на 
$\pi$ та на площині правого зображення --- на прямій їх перетину. 
Вона має проходити через праву епіполь $e'$, оскільки точки $O$ проектується в
епіполь $e'$. Назвемо зпроектовану пряму прямою $l'$. На ній і повинна лежати 
точка $x'$. 

Назва прямої $l'$ --- епіполярна пряма правого зображення. Аналогічно знаходимо 
пряму $l$ --- епіполярну пряму лівого зображення, яка проходить через ліву 
епіполь $e$. Епіполярна площина $\pi$ перетинає площини лівого та правого 
зображеннь та визначає відповідність між епіполярними лініями $l$ та $l'$.

Можна помітити, що в незалежності від вибранної точки $X$, епіполярна площина 
буде проходити через епіполі. Звідси і епіполярні лінії завжди будуть перетинати 
епіполі. Адже епіполь --- це проекція центра камери на зображення. Таким чином 
всі епіполярні лінії перетинаються в епіполі. Кажуть, що епіполярні лінії
збігаються до епіполі. Оскільки чим ближче точка сцени до центра камери, тим 
ближче проекція даної точки до епіполярної точки.


\section{Матриця гомографії та фундаментальна матриця}
Завдяки лінійності проектування в камері-обскурі, відношення між точкою $x$ 
на лівому зображенні та відповідною епіполярною лініею $l'$ теж лінійне. 
Фундаментальна матриця $F \in \mathbb{R}^{3\times3}$ описує це відношення з
точністю до константи формулою
\begin{equation}
l' = F \cdot x.
\end{equation}
Звідси дуже легко знаходимо наступне епіполярне обмеження. Оскільки точка $x'$, 
яка відповідає $x$, лежить на прямій $l'$, то
\begin{equation}
{x'}^T \cdot F \cdot x = 0.
\end{equation}
Епіполі $e$ та $e'$ утворюють праве та ліве нульові ядра фундаметальної матриці 
$F$. Маємо 
\begin{gather}
F \cdot e = 0 
\\
F^T \cdot e' = 0.
\end{gather}

Введемо поняття матриці гомографії. Матриця гомографії --- це матриця 
$H \in \mathbb{R}^{3\times3}$, яка пов'язує проекцію певної точки сцени на ліве
зображення, з її проекціею на праве зображення. ЇЇ зв'язок з фундаментальною
матрицею $F$ визначається наступним співвідношенням \cite{LUONG1996193}
\begin{equation}
F = {[e']}_\times \cdot H,
\end{equation}
де кососиметрична матриця векторного добутку
\begin{equation}
{[e']}_\times \:\:\: = \:\:\:
\left[
\begin{matrix}
0 & -{e'}_z & {e'}_y\\
{e'}_z & 0 & -{e'}_x\\
-{e'}_y & {e'}_x & 0\
\end{matrix}
\right].
\end{equation}

З нерівності (1.5) нескладно показати, що $\operatorname{rank}(F) = 2$. Оскільки
$\operatorname{rank}({[e']}_\times) = 2$, а $\operatorname{rank}(H) = 3$,

//Чи розписувати, чому це правда?\\
Введемо іншу властивість фундаметнальної матриці 
\begin{equation}
F = C \cdot {[e']}_\times \cdot {[e']}_\times \cdot F,
\end{equation}
де $C$ --- деяка константа, а ${[e']}_\times$ --- матриця векторного добутку.

З формул (1.5) та (1.7) маємо, що існує деяка матриця
\begin{equation}
S = C \cdot {[e']}_\times \cdot F,
\end{equation}
що має схожі властивості з матрицею гомографії $H$. А саме --- вона пов'язує 
проекцію певної точки сцени на ліве зображення, з деякою точкою, яка лежить на 
одній епіполярній прямій з проекціею на праве зображення. Це найважливіша 
властивість матриці $S$ для подальшої роботи.
\\//Розписати, що за схожі властивості, відмінність S і H\\


\section{Ректифікація зображеннь}
Ректифікацією зображень називається процес горизонтального випрямлення 
епіполярних ліній зображення та винесення епіполі на нескінченність.
Ректифікація трансформує вхідні зображення так, наче вони були зроблені 
тільки з горизонтальним зміщенням. Як наслідок: всі епіполярні лінії --- 
горизонтальні. Отже, щоб ректифікувати зображення треба зробити 
епіполярні лінії паралельними осі $X$, що означає  --- винести епіполь 
зображення на нескінченність.
\begin{equation}
	R_r \cdot e' = {[f_r \:\:\: 0 \:\:\: 0]}^T,
\end{equation}
де $R_r$ --- матриця ректифікації правого зображення, $e'$ --- права епіполь, а 
$f_r$ - деяка константа.

Аналогічно і для лівого зображення
\begin{equation}
	R_l \cdot e = {[f_l \:\:\: 0 \:\:\: 0]}^T,
\end{equation}
де $R_l$ --- матриця ректифікації лівого зображення, $e$ --- ліва епіполь, а 
$f_l$ - деяка константа.

Задача ректифікації і полягає в пошуку таких $R_r$ та $R_l$, щоб виконувалися 
умови (1.9) та (1.10). Задача ж ректифікації стеретеопари полягає у отриманні 
ідеальної стереопари. В ідеальній стереопарі відповідні горизонтальні 
епіполярні лінії ректифікованих правого та лівого зображень мають бути на 
одному рівні по вертикалі. А саме --- мати однакові $Y$ координати. Зробити 
з вхідної стереопари ідеальну --- головна задача цієї роботи.

Виникає питання: Як оцінювати якість ректифікації стереопари; ідеальність 
стереопари? Питання риторичне. Оскільки зазвичай після завдання ректифікації 
стереопари, пару зображеннь зміщують, то оцінка відбувається по точності 
зміщення двох зображень. Під зміщенням мається на увазі різниця в 
місцезнаходженні відповідних точок стереопари при поеднанні двох 
ректифікованих зображень. Ми будемо використовувати суму різниць квадратів 
відповідних точок після ректифікації стереопари як основну метрику.



\chapter{Алгоритм ректифікації стереопари}



\section{Постановка задачі}
Задача ректифікації стереопари полягає у перетворенні вхідної стереопари в
ідеальну стереопару. 
\\// Питання чи достатньо матриці М для ідеальної стереопари?

Метод полягає у ретифікації спочатку правого зображення, винесенням правої 
епіполі на нескінченність. Потім лівого. Ліве зображення ретифікується за 
допомогою матриці ректифікації правого зображення $R_r$ та матриці $S$, яка 
містить фундаментальну матрицю. Задля оцінки ректифікації, вирішується одна 
оптимізаційна задача для знаходження фінальної матриці $R_l$.

Головна відмінність данної роботи від попередніх --- у матриці $S$. Метод 
Hartley\cite{DBLP:journals/ijcv/Hartley99} використовував замість неї матрицю 
гомографії $H$ та вирішував декілька оптимізаційних задач. Метод 
Малона\cite{Mallon2005ProjectiveRF} аналогічно Hartley знаходить трохи іншу 
матрицю перетворення, але так само вирішує декілька оптимізаційних задач. Наш 
запропонованій метод шукає матрицю ректифікації правого зображення за допомогою 
фундаментальної матриці $F$. А використовуючи властивість матриці $S$, метод 
вирішує тільки одну оптимізаційну задачу вздовж $X$ координати зображень.



\section{Пошук фундаментальної матриці}
\subsection{Знаходження ключових точок}
Ключові точкі --- це характерні координати зображення разом з їх дескрипторами.
Дескриптори --- це вектори певної довжини, які максимально точно описують деяку 
точку зображення. По суті, ключові точки є ознакою зображення.

Є різні методи вилучення ключових точок із зображення 
SIFT\cite{10.1023/B:VISI.0000029664.99615.94}, SURF\cite{Bay2006SURFSU}, 
ORB\cite{6126544} та інші. Ми використовуємо традиційний SIFT алгоритм. 
Завдяки ньому отримаємо перелік координат та векторів з закодованими 
ознаками відповідної координати зображення. Тепер переліки двох зображень 
стереопари треба співставити.


\subsection{Співставлення ключових точок}
Ми знайшли ключові точки окремо для кожного зображення. Співставимо їх. Тим самим
дізнаємось, які з них присутні на обох зображеннях. Та дізнаємось різницю в 
росташуванні відповідних точок на різних зображеннях. Це потрібно для подальшого 
знаходження фундаментальної матриці стереопари.

Ми використовуємо метод співставлення FLANN\cite{Muja09fastapproximate}. Він 
сортує найкращі потенційні співставлення дескрипторів відповідних точок. Як
результат отримуємо відфільтрований перелік координат ключових точок лівого 
зображення та відповідний їм перелік координат ключових точок правого зображення.


\subsection{Знаходження фундаментальної матриці}
Фундаментальна матриця $F$ показує лінійне відношення між точкою на лівому 
зображенні та відповідною епіполярною ліноєю на правому зображенні. Тому маючи 
перелік відповідних ключових точок стереопари можна знайти матрицю $F$. 

Існують різні методи для знаходження фундаментальної матриці за допомогою 
відповідних точок стереопари: Seven-Point\cite{Hartley94projectivereconstruction},
Eight-Point\cite{LonguetHiggins1981ACA}, RANSAC\cite{10.1145/358669.358692} 
та інші. Ми вибрали стандартний RANSAC. На виході маємо фундаментальну матрицю
$F$ з лівого у праве зображення нашої стереопари.


\subsection{Знаходження правої епіполі}
Знайдемо праву епіполярну точку. За (1.4) --- це праве ядро фундаментальної 
матриці $F$. Транспонуємо матрицю. Далі використовуємо SVD\cite{1102314}, 
де знаходимо ортонормований базис нульвого простору отриманної матриці. 
\begin{equation}
e' = {[{{e'}_n}_x \:\:\: {{e'}_n}_y \:\:\: {{e'}_n}_z]}^T.
\end{equation}

//Чому не втратимо загальності? Чи треба це пояснити?\\
Не втрачаючи загальності, можемо спростити розрахунки та пронормувати епіполі.
Зробимо нормування по $Z$ координаті (також реалізований варіант нормування 
по нормі епіполі). Тоді права епіполь матиме вигляд
\begin{equation}
e' = {
\begin{bmatrix}
\frac{{{e'}_n}_x}{{{e'}_n}_z} & \frac{{{e'}_n}_y}{{{e'}_n}_z} & 
\frac{{{e'}_n}_z}{{{e'}_n}_z}
\end{bmatrix}
}^T = {[e'_x \:\:\: e'_y \:\:\: 1]}^T.
\end{equation}

Аналогічно ліва епіполь матиме вигляд
\begin{equation}
e = {[e_x \:\:\: e_y \:\:\: 1]}^T.
\end{equation}

У вигляді кососиметричної матриці для векторного добутку
\begin{equation}
{[e']}_\times \:\:\: = \:\:\:
\left[
\begin{matrix}
0 & -1 & {e'}_y\\
1 & 0 & -{e'}_x\\
-{e'}_y & {e'}_x & 0\
\end{matrix}
\right]
\end{equation}
та
\begin{equation}
{[e]}_\times \:\:\: = \:\:\:
\left[
\begin{matrix}
0 & -1 & {e}_y\\
1 & 0 & -{e}_x\\
-{e}_y & {e}_x & 0\
\end{matrix}
\right]
\end{equation} 



\section{Матриця ректифікації правого зображення}
\subsection{Зміщення центру правого зображення}
Введемо розміщення правої камери у просторі. Її зображення нехай лежить на
площині $z=1$, а центр камери --- в нулі координат. Відразу помітимо, що вектор
правої епіполі, який ми знайшли раніше, може дивитися в сторону протилежну 
площині $z=1$ (Це не вплине на якість результуючих зображень, а лише зеркально 
переверне їх).

//Чи буде місцезнаходження центру зображення впливати на результат, чи тільки на
зміщення результату?\\
Центр правого зображення має бути перпендикулярний до площини $z=1$ і 
знаходиться в $[0 0 1]$. Але за замовчуванням центр зображення розташований 
у лівому верхньому кутку зображення. Саме тому введемо матрицю зміщення 
$T \in \mathbb{R}^{3\times3}$. Вона буде зміщувати лівий верхній куток 
правого зображення з точки $[0 0 1]$, так щоб в $[0 0 1]$ опинився 
центр правого зображення.
\begin{equation}
T \:\:\: = \:\:\:\left[
\begin{matrix}
1 & 0 & -c_x\\
0 & 1 & -c_y\\
0 & 0 & 1\
\end{matrix}
\right],
\end{equation}
де $c_x = \frac{w}{2}; \:\:\: c_y = \frac{h}{2}$, а $w$ і $h$ --- розміри 
правого зображення.

Також знайдемо обернену до $T$, вона нам знадобиться для фінального зміщення
отриманого результату назад.
\begin{equation}
T^{-1} \:\:\: = \:\:\:\left[
\begin{matrix}
1 & 0 & c_x\\
0 & 1 & c_y\\
0 & 0 & 1\
\end{matrix}
\right],
\end{equation}
де $c_x = \frac{w}{2}; \:\:\: c_y = \frac{h}{2}$, а $w$ і $h$ --- розміри 
правого зображення.

Права епіполь теж лежить на правому зображенні, тож її теж треба змістити.
Перевіремо куди напрямлена знайдена права епіполь. Якщо її $Z$ координата
від'ємна, то домножимо матрицю зміщення на мінус: $T = -T$. Цим ми 
дзеркально трансформуємо результат. 

Подіємо $T \cdot e'$. Надалі при використанні $e'$, ми будемо мати на увазі 
праву епіполь зі зміщенням $T \cdot e'$:
\begin{equation}
e' = T \cdot e' = {
\begin{bmatrix}
e'_x - c_x & e'_y - c_y & 1
\end{bmatrix}
}^T.
\end{equation}


\subsection{Занулення $Y$ координати правої епіполі}
Тепер, коли ми розуміємо розміщення правого зображення у просторі, можна 
ректифікувати зображення. Винесемо праву епіполь на нескінченність --- 
${[f_r \: 0 \: 0]}^T$, де $f_r$ --- деяка константа. Почнемо з занулення $Y$ 
координати. Для цього введемо ортонормовану праву матрицю повороту $R$, 
таку, що 
\begin{equation}
R \cdot e' = {
\begin{bmatrix}
(R \cdot e')_x & 0 & 1
\end{bmatrix}
}^T.
\end{equation}

Введемо вектор $k = {[0 \: 0 \: 1]}^T$ --- один з базисних векторів простору.
Побудуємо $R$ таким чином
\begin{equation}
R \:\:\: = \:\:\:\left[
\begin{matrix}
\frac{{(k \times (e' \times k))}^T}
{\left \| (k \times (e' \times k)) \right \|}\\
\frac{{(k \times e')}^T}{\left \| (k \times e') \right \|}\\
k^T\
\end{matrix}
\right].
\end{equation}

Де $R$ не чіпає $Z$ координату; зануляє $Y$ координату: 
${(k \times e')}^T \cdot e' = 0$; а $X$ координата --- векторний добуток
двох інших, таким чином, щоб матриця залишалася правосторонньою.

Введемо константу $d = \sqrt{{e'_x}^2 + {e'_y}^2}$. Cпростимо матрицю повороту $R$
\begin{equation}
R \:\:\: = \:\:\:\left[
\begin{matrix}
\frac{1}{d} \cdot {(k \times (e' \times k))}^T\\
\frac{1}{d} \cdot {(k \times e')}^T\\
k^T\
\end{matrix}
\right].
\end{equation}

У схожій статті \cite{Nechaieva} будували аналогічну матрицю повороту, але 
лівосторонню. Ця різниця спричинила появу крайових випадків, коли даний 
метод ректифікації не спрацьовував. В цій роботі цих недоліків нема і 
метод працює у всіх випадках.


\subsection{Занулення $Z$ координати правої епіполі}
Тепер, коли права епіполь має вигляд ${[(R \cdot e')_x \: 0 \: 1]}^T$ 
занулемо її $Z$ координату. Введемо матрицю трансвекції $G$, таку, що
\begin{equation}
G \cdot 
\begin{bmatrix}
(R \cdot e')_x & 0 & 1
\end{bmatrix}
 = {
\begin{bmatrix}
(R \cdot e')_x & 0 & 0
\end{bmatrix}
}^T.
\end{equation}

Побудуємо $G$, таким чином 
\begin{equation}
G \:\:\: = \:\:\:\left[
\begin{matrix}
1 & 0 & 0\\
0 & 1 & 0\\
\frac{-1}{{(R \cdot e')}_x} & 0 & 1\
\end{matrix}
\right].
\end{equation}

Можемо спростити скаляр ${(R \cdot e')}_x = d$. Тоді маємо 
\begin{equation}
G \:\:\: = \:\:\:\left[
\begin{matrix}
1 & 0 & 0\\
0 & 1 & 0\\
\frac{-1}{d} & 0 & 1\
\end{matrix}
\right].
\end{equation}

Підсумовуючи, отримали праву епіполь у вигляді ${[d \: 0 \: 0]}^T$. Саме такий 
вигляд і приймає права епіполь після дії на неї правої матриці ректифікації
за (1.9). Звідси ${[f_r \: 0 \: 0]}^T = {[d \: 0 \: 0]}^T$, де $f_r = d$.


\subsection{Об'єднаємо всі перетворення у праву матрицю ректифікації}
Завдання було знайти таку матрицю ректифікації правого зображення, що 
виносить на некінченність праву епіполь. 
\begin{equation}
	R_r \cdot e' = {[f_r \:\:\: 0 \:\:\: 0]}^T,
\end{equation}
де $R_r$ --- матриця ректифікації правого зображення, $e'$ --- права епіполь, а 
$f_r$ - деяка константа.

Об'єднавши всі попередні перетворення отримаємо праву матрицю ректифікації яка 
задовальняє умову (1.9)
\begin{equation}
	R_r = T^{-1} \cdot G \cdot R \cdot T,
\end{equation}
де $T$ --- матриця зміщення центру зображення, $R$ --- права ортонормована
матриця повороту, $G$ --- матриця трансвекції.

\begin{equation}
\begin{array}{l}
R_r  \:\:\: = \:\:\:
\left[
\begin{matrix}
1 & 0 & cx\\
0 & 1 & cy\\
0 & 0 & 1\
\end{matrix}
\right]
\cdot
\left[
\begin{matrix}
1 & 0 & 0\\
0 & 1 & 0\\
\frac{-1}{d} & 0 & 1\
\end{matrix}
\right]
\cdot
\left[
\begin{matrix}
\frac{1}{d} \cdot {(k \times (e' \times k))}^T\\
\frac{1}{d} \cdot {(k \times e')}^T\\
k^T\
\end{matrix}
\right]
\cdot
\left[
\begin{matrix}
1 & 0 & -cx\\
0 & 1 & -cy\\
0 & 0 & 1\
\end{matrix}
\right] \:\:\: =
\\\\
= \:\:\:
\left[
\begin{matrix}
1 & 0 & cx\\
0 & 1 & cy\\
0 & 0 & 1\
\end{matrix}
\right]
\cdot
\left[
\begin{matrix}
\frac{1}{d} \cdot {(k \times (e' \times k))}^T\\
\frac{1}{d} \cdot {(k \times e')}^T\\
k^T - \frac{1}{d^2} \cdot {(k \times (e' \times k))}^T\
\end{matrix}
\right]
\cdot
\left[
\begin{matrix}
1 & 0 & -cx\\
0 & 1 & -cy\\
0 & 0 & 1\
\end{matrix}
\right].
\end{array}
\end{equation}
Таким чином ми знайшли $R_r$, що задовольняє (1.9).



\section{Матриця ректифікації лівого зображення}
\subsection{Проміжна матриця ректифікації}
Головна складність задачі не ректифікувати ліве зображення, а знайти таку 
матрицю ректифікації лівого зображення, щоб разом з матрицею ректифікаціїї
правого зображення перетворювала стереопару у ідеальну стереопару.

Основна ідея пошуку такої матриці ректифікації лівого зображення полягає 
у тому, щоб перенести певною матрицею точки лівого зображення у відповідні 
точки правого зображення. А потім, за допомогою вже відомої матриці ректифікації 
правого зображення, ректифікувати перенесені точки лівого зображення. 
Матрицею, що переносить точки лівого зображення у відповідні точки правого 
зображення, може бути матриця гомографії чи інші подібні матриці. 
В цій роботі ми використовуємо матрицю $S = C \cdot {[e']}_\times \cdot F$, 
описану в пункті 1.4. 

Переваги використання саме матриці $S$ —- у простоті її розрахунку. У 
порівнянні з матрицею гомографії, щоб знайти фундаментальну матрицю нам 
потрібно лише від 8ми ключових точок. Звісно, простота розрахунку має свої 
недоліки. $\operatorname{Rank}(S)$ дорівнює двом, а не трьом (як у матриці 
гомографії). Це означає, що матриця S не переносить точки лівого зображення 
у відповідні точки правого зображення. Але нам цього і не треба. Достатньо 
лише властивості матриці S: вона пов'язує проекцію певної точки сцени на 
ліве зображення, з деякою точкою, яка лежить на одній епіполярній прямій 
з проекціею на праве зображення. Цього достатньо, щоб отримати ідеальну 
стереопару. Головне, що точки лівого зображення, перенесені матрицею $S$,
залишалися на відповідних епіполярних лініях правого зображення.

Введемо бажану проміжну матрицю ректифікації лівого зображення, створену 
за допомогою матриці S
\begin{equation}
	M = R_r \cdot S
\end{equation}

//Перевірити точність, винесенням на нескінченність лівої епіполі?\\
Тепер маємо матрицю ректифікації $M$, яка задовольняє умови (1.10) та 
разом з матрицею $R_r$ створює ідеальну стереопару. Але знов постає питання: 
Як перевірити та оцінити ректифікацію стереопари? Порахуємо суму різниць
квадратів відповідних точок після ректифікації лівого та правого зображень. 
Проте зараз проміжна матриця ректифікації лівого зображення $М$ незрозумілим 
чином деформує ліве зображення. І сума різниць квадратів не дасть ніякої 
об'єктивної оцінки. Трансформуємо матрицю $M$ таким чином, щоб можна було 
оцінити якість ректифікації стереопари сумою різниці квадратів.


\subsection{Оптиміізаційна задача для подальшої оцінки ректифікації}
Розберемося як саме проміжна матриця ректифікації $M$ перетворює ліве 
зображення. Для початку зазначемо, що $M$ переносить ліве зображення у 
площину, компланарну до площини правого зображення. Чому не саме у площину 
правого зображення? Бо матриця $S$ єдина з точністю до константи $C$. 
Тому після дії матриці $M$ на ліве зображення і матриці $R_r$ на праве 
зображення ми нормуємо отримані зображення по $Z$ координаті. Так ми 
проектуємо їх на одну площину $z=1$, де вже можемо якось порівнювати
$X$ та $Y$ координати.

Матриця $M$ ректифікує ліве зображення і робить всі епіполярні лінії
горизонтальними. Також $M$ разом з $R_r$ вирішують задачу ідеальної 
стереопари --- всі відповідні епіполярні лінії мають однакові $Y$ координати.
Отже немає сенсу порівнювати $Y$ координатb на площині $z=1$ --- вони 
однакові.

Єдине, що не зрозуміло --- як розташовані точки ректифікованого лівого
зображення відносно осі $X$. А це треба знати для їх подальшого порівняння
з відповідними точками ректифікованого правого зображення. По суті ми 
можемо змінити $X$ координату матриці $M$ як завгодно, при цьому властивість 
ректифікації (1.10) та задача ідеальності стереопари не будуть порушені. 
Тому для хорошої оцінки ректифікації стереопари підкоректуємо $M_x$. 

Введемо параметри $a, b, c \in \mathbb{R}$ такі, що $M_x = {[a \: b \: c]}^T$.
Оптимізуємо їх таким чином, щоб відповідні точки після ректифікації 
лівого та правого зображень були якнайближче. Зробимо це за допомогою 
наступної оптимізаційної задачі
\begin{equation}
\sum_{\left< x, x'\right> \in \theta}{\left (  
{\begin{bmatrix}
a \\ b \\ c
\end{bmatrix}}^T
\cdot
\frac{x}{{(M \cdot x)}_z}
-
\frac{{(M \cdot x')}_x}{{(M \cdot x')}_z}
\right )}^2
\to \underset{a, b, c \in \mathbb{R}}{min},
\end{equation}
де $\theta$ --- набір відповідних точок стереопари.

Отримали фінальну матрицю ректифікації лівого зображення стереопари
\begin{equation}
R_l \:\:\:  =  \:\:\:
\left[
\begin{matrix}
a & b & c\\
& M_y &\\
& M_z &\
\end{matrix}
\right].
\end{equation}

Залишилося тільки проаналізувати практичні результати алгоритму.

